\documentclass{report}
\usepackage{amsmath}
\usepackage{amssymb}
\begin{document}

\title{Adaptive Cruise Control System Model}
\author{Your Name}
\date{\today}

\maketitle

\tableofcontents{}

\chapter{Part 1 (a)}
\section{Introduction}
This part presents the derivation of a system model for an Adaptive Cruise Control (ACC) system. The model is based on the longitudinal dynamics of a vehicle and the dynamics of vehicle actuation. The model also considers the travel resistance which includes air drag, rolling resistance, acceleration resistance, and grading resistance.

\section{Vehicle Dynamics Model}
The longitudinal dynamics of a host vehicle is given by the equation:
\begin{equation}
    m\dot{v}_h = m a_f - r_{travel}
\end{equation}
where \(m\) is the vehicle mass, \(v_h\) is the vehicle speed, \(a_f\) is the traction force converted to acceleration, and \(r_{travel}\) is the travel resistance.

The dynamics of a vehicle actuation including engine, transmission, or brake has nonlinear characteristics. The input/output relationship of the actuation dynamics is described as an ordinary differential equation:
\begin{align}
    \dot{x}_f & = f_{act}(x_f, u) \\
    a_f       & = h_{act}(x_f)
\end{align}
where \(x_f \in \mathbb{R}^{n_f}\), \(u \in \mathbb{R}\) are respectively the state, the input of the actuation system. \(u\) is an acceleration command, i.e., a control input calculated by adaptive cruise controller. The output of the system is \(a_f\).

The travel resistance \(r_{travel}\) contains several factors to resist its motion. A model of the resistive force is expressed as
\begin{equation}
    r_{travel} = r_{air} v_h^2 + r_{roll}(\dot{v}_h) + r_{accel} \dot{v}_h + r_{grad}(\theta)
\end{equation}

\section{State-Space Model for ACC System}
To build a plant model for the ACC system design, two state variables are defined: inter-vehicle distance following error \(\Delta d = d - d_r\) and velocity following error \(\Delta v = v_p - v_h\). The \(d_r\) is determined based on the constant time headway policy given by
\begin{equation}
    d_r = T_{hw}v_h + d_0
\end{equation}
where \(T_{hw}\) is the constant time headway and \(d_0\) is the stopping distance for safety margin.

Let us define the state variables of the plant as \(x = [x_1 \quad x_2 \quad x_3^T]^T \in \mathbb{R}^{2+n_f}\) with \(x_1 = \Delta d\), \(x_2 = \Delta v\) and \(x_3 = x_f\). Then, the state-space model is formulated as
\begin{align}
    \dot{x} & = f(x, u) + Gv + Hw \\
    y       & = Cx + Jv
\end{align}
where
\begin{align*}
    f(x, u) & = \begin{bmatrix}
                    x_2 - T_{hw}x_3 \\
                    -h_{act}(x_f)   \\
                    f_{act}(x_f, u)
                \end{bmatrix}, \\
    G       & = \begin{bmatrix}
                    T_{hw}/m \\
                    1/m      \\
                    0
                \end{bmatrix},           \\
    H       & = \begin{bmatrix}
                    0 \\
                    1 \\
                    0
                \end{bmatrix},           \\
    C       & = \begin{bmatrix}
                    1 & 0 & 0 \\
                    0 & 1 & 0 \\
                    0 & 0 & 1
                \end{bmatrix},           \\
    J       & = \begin{bmatrix}
                    0 \\
                    0 \\
                    -1/m
                \end{bmatrix}
\end{align*}
where \(u \in \mathbb{R}\) and \(y = [\Delta d \quad \Delta v \quad \dot{v}_h]^T \in \mathbb{R}^3\) are the input and output of the plant and \(v = r_{travel}\) and \(w = \dot{v}_p\) represents disturbances into the plant.

\end{document}
